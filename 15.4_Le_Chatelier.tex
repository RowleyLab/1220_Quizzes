\documentclass[11pt, letterpaper]{memoir}
\usepackage{HomeworkStyle}
\geometry{margin=0.75in}

\begin{document}
	\begin{center}
		{\large	Quiz 15.4 -- Le Ch\^atelier's Principle}
	\end{center}
	{\large Name: \rule[-1mm]{4in}{.1pt} 
	
	\subsection*{Question 1}
	Consider the reaction: \ch{NH3(aq) + H2O(l) <=> NH4^+(aq) + OH^-(aq)}
	
	\noindent Tell which way the reaction will shift in response to the following changes:
	\begin{itemize}
		\item Remove \ch{OH^-} through a precipitation reaction with \ch{Sr^{2+}}
		
		~
		\item Add ammonium nitrate salt to the solution
		
		~
		\item Place the reaction solution in contact with high pressure \ch{NH3(g)}
		
		~
		\item Dilute the reaction solution to $4$ times its initial volume
		
		~
		\item Boil away most of the solvent
	\end{itemize}
	
	\vspace{1em}
	\subsection*{Question 2}
	Consider the reaction: \ch{C3H8 (g) + 5 O2(g) <=> 3 CO2(g) + 4 H2O(g)} \hspace{1em} $\Delta H_{rxn}=-2044~\dfrac{kJ}{mol}$
	
	\noindent This reaction is highly product-favored. If you wanted to produce \ch{C3H8} from \ch{CO2} and \ch{H2O}, what pressure and temperature conditions should you use? (ignore kinetic considerations)
	
	\vspace{4em}
	\subsection*{Question 3}
	A certain gas reaction has colorless reactants and dark brown products. This reaction has reached equilibrium in a transparent piston with a movable head. The reaction begins colorless, but turns brown as you reduce the volume by pressing down on the piston head with high pressure. You then decrease the temperature while maintaining the reduced volume and the reaction again turns colorless. What can you say about the stoichiometry and enthalpy of this reaction?
		
	\newpage
	\newgeometry{margin=1.25in}
	\pagestyle{empty}
	\addtocounter{page}{-1}
	\section*{\emph{Sonnet 18: Shall I compare thee to a summer’s day?}}
	\paragraph{By William Shakespeare}~
	\begin{verse}
		Shall I compare thee to a summer’s day?\\
		Thou art more lovely and more temperate:\\
		Rough winds do shake the darling buds of May,\\
		And summer’s lease hath all too short a date;\\
		Sometime too hot the eye of heaven shines,\\
		And often is his gold complexion dimm'd;\\
		And every fair from fair sometime declines,\\
		By chance or nature’s changing course untrimm'd;\\
		But thy eternal summer shall not fade,\\
		Nor lose possession of that fair thou ow’st;\\
		Nor shall death brag thou wander’st in his shade,\\
		When in eternal lines to time thou grow’st:\\
		\hspace{0.5em} So long as men can breathe or eyes can see,\\
		\hspace{0.5em} So long lives this, and this gives life to thee.
	\end{verse}
\end{document}
